\subsection{Overview}
In this whole chapter the architecture design process is described, and it contains all the significant
architecture decisions. 
Therefore, first of all, in this section an high-level and very general architecture for the TrackMe 
project is presented and discussed.  

\begin{figure}[H]
\includegraphics[width=\linewidth]{Images/highlevelcd.pdf}
\caption{ High-level component diagram }
\label{fig:highlevelcomponentdiagram}
\end{figure}

As a reference architecture, the microservices one has been considered: this decision will be later discussed. \\
From the picture, it is evident that the clients can access the services by means on API gateway, that will provide authentication 
of the users, and forwards the requests to a router. The router will query a service registry, in order to know where are located instances 
of the demanded service: once that it retrieves this information, it will be able to send and load balance requests.
Of course, a new service instance needs to register to the service registry. \\
Another relevant information is that the project can be ideally split, at an high level, into two applications: one for the users and one for the third parties customers: they both interact with the router, that will access all the
microservices that compose the core functionality of the business.
More specifically, the "microservices" block includes Data4Help and Track4Run functions, while the goals
of AutomatedSOS service are reached locally and directly by means of the user application, that, when
needed, will perform a call with the help of the SIM service. The user application also needs to
communicate with a device that is able to collect data that regards his health: more specifically, this
instrument will send data toward the user application autonomously, without being asked. Examples of 
this machines are smartrings and smartwatches. The data regarding the position of the user is, instead, retrieved via GPS's API. \\
Note, also, that the microservices need to keep some data persistent, and, thus access to a DBMS, by means
of its API. Moreover, some services need also to communicate and share information: this is achieved via the message queue, that is
charged of receiving and forward message to the interested parts. \\

