\subsection{Other design decisions}
\subsubsection{Frameworks}
The chosen and described architectural style, leads to some choices in the framework that will be used for the development and the deployment. 
In particular, to implement a microservices architecture, a microservices chassis framework is needed, in order to have the best support 
in the developing phase. 
An option here is to adopt both Spring Boot (for what concerns the microservices) and Spring Cloud (that facilitates the set up of distributed
system software). They are widely adopted and greatly supported by the community; also many guides and articles are available on the internet. \\

\subsubsection{Asynchronous messages}
RabbitMQ, that is a message broker software, is used to implement the exchange of messages among services. This choice is motivated 
by the following facts: it enhances delivery and order guarantee, redundancy, decoupling and scalability. 
Finally, it is well supported in Spring boot. 

\subsubsection{Automated SOS encryption and data transfer}
Another important issue is regarding the communication between the user hardware device and the user application. This is encrypted with
AES-128. \\
The communication happens by means of bluetooth: the version 4.0 + LE, that is available from 2010, provides some nice features for the
project, because it grants enough bandwidth (1Mbps when the low energy mode is enabled) for transmitting the data, and it is also a low
energy consumer. Furthermore, there is no need of internet connection (with the limits of the network coverage): the devices just need to be
close, but this is a totally rational assumption, given the context of the application. 
Note that this version ensures a good compatibility with a large set of devices. 

\subsubsection{Android application} 	 
The user application will be developed for the Android operative system, at least in a first moment. This is due to the criticality of the
automatedSOS service. Assuming that the application will be deployed with Italy as the first market target, this choice covers the biggest
part of the market.

\subsubsection{Choice of external services}
Google maps platform is the choice adopted for the external service regarding the maps. This decision is motivated by its reliability;
moreover, it is kept under update and it is well supported by the community. Furthermore, if in the client it is necessary to show some maps,
as it is, the one provided by Google are thought to be the most trusted by the users. 


\subsubsection{JSON}
REST API will be implemented using JSON as a protocol for exchanging data. \\
Compared to XML, it is less verbose and JSON packets require less size. Furthermore, it is well supported
by Spring Boot. 

\subsubsection{Group Request Filtering}
Up until now, group requests were more or less abstract: there was no definition of what group requests could possibly get from 
data. Therefore, here it is defined what group requests return. The only outcomes they give are  aggregated data, i.e. count 
of tuples, sum of values, average of values or other aggregate operator available (e.g. in SQL). Another thing to clarify is that 
the tuples of the result from a group request have to satisfy a certain constraint requested by the petitioner. Consequently, 
the following filters are the ones available:
\begin{itemize}
\item Filtering by birthplace
\item Filtering by age
\item Filtering by a combination of birth year/month/day
\item Filtering by the GPS position data
\item Filtering by the health status
\item Filtering by date and time
\item Filtering by name or surname (not both of them, otherwise it will compromise privacy data)
\end{itemize}