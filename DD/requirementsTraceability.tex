\section{Requirements traceability}
\subsection{Functional requirements}
In this section, the requirements and goals specified into the \textit{Requirements Analysis and Specification Document} are discussed. 
The following list provides the design components that allows to satisfy the requirements and goals: 
\begin{itemize}
\item[{[G1]}] Allow a user to access its own data. Requirements([R10])
	\begin{enumerate}
	\item Account service.
	\item Share data service.
	\end{enumerate}
As already mentioned in the \textit{Component view} chapter, the account service provides all the user session functions, like the log in into
the application, fundamental for accessing user's data. 
Furthermore, the account service allows a user to perform all the other possible actions, since the user must be subscribed and logged into
the application: for this reason, this is always necessary for reaching all goals, but for more legibility it is described only in this first
point. 
Share data service, instead, is responsible for monitoring the user's information, and allows to download the user data from the server.

\item[{[G2]}] Allow a user to contribute to data sharing by providing information about his location and health status. Requirements([R11])
	\begin{enumerate}
	\item Account service.
	\item Share data service.
	\end{enumerate}
Share data service allows a user to upload its data through SendDataService. 
Of course, this is possible only for a user who is logged in. 

\item[{[G3 \& G4]}] Once the health parameters of a user have been observed below the threshold for the first time after one hour, an ambulance is sent to the user location. 
The time experienced between the moment in which the health parameters of a subscribed user are observed below the threshold and the time in which the emergency point is contacted is equal or less than 5 seconds. Requirements([R12]-[R13]-[R14]-[R14]-[R15]-[R16]-[R17])
	\begin{enumerate}
	\item Account service.
	\item GPS sensor.
	\item User HW Device.
	\item SIM services.
	\end{enumerate}
This is one of the most critical feature. 
The user hardware device provides the heath status of the owner and, when at least one parameter is observed below the threshold, a reading
from the GPS sensor is performed, if needed. 
With this data (i.e. GPS coordinate and health status), the SIM service is able to call the emergency room and provide them all the necessary 
information. 
Note that for guaranteeing this functionality the connection between the mobile device and the sensors must be present, and the user must be
in an area, where is possible to perform a call.

\item[{[G5]}] Allow a user to participate in a run managed by third parties, as an athlete, if all starting conditions are satisfied. Requirements([R18]-[R19]-[R20]-[R21]-[R22])
	\begin{enumerate}
	\item Account service.
	\item Race service.
	\end{enumerate}
The service that allows to achieve this goal is the Race service. 
This service, as already specified, makes possible for an athlete to enroll in a run by showing a list of the available race.
Furthermore, it manages the race status and all the possible actions that a user or the third party can perform on it. (i.e. close it, set
date, subscribe, etc...) 

\item[{[G6]}] Allow spectators (i.e. user) to see on real-time the "correct" positions of all athletes taking part in a run, with at most 15 meters of radius error.
Requirements([R23]-[R24])
	\begin{enumerate}
	\item Account service.
	\item Spectator service.
	\end{enumerate}
As it can be deduced from the name, Spectator service regards the dynamic part of a race and allows spectators to see on real-time the 
athletes in a run.
The service receiving position data from the participants and shares it with the spectators by showing on the map. 
	
\item[{[G7]}] The maximum time to accept an individual request from any third party is 30 days; after that, the request will expire. Requirements([R25])
	\begin{enumerate}
	\item Account service.
	\item Individual request service.
	\end{enumerate}
All the constrictions on the individual requests are handled by the Individual request service, which receives the requests from the third
party customers. 
This service is charged of calculating the time between the date when the request is posted, and the current date: after that, it decides if
the request is expired or not, and, eventually, updates the status.

\item[{[G8 \& G9]}] Allow a user to accept or refuse a request from third parties. Allow a user to block requests made by a specific third party and all the pending requests will be refused. Requirements([R26]-[R27]-[R28]-[R29])
	\begin{enumerate}
	\item Account service.
	\item Individual request service.
	\end{enumerate}
Always the Individual request service manages the accepting or the refusing of the request or the blocking of one by showing a list of the pending requests. 	
	
\item[{[G10]}] Allow spectators and runners to see the leaderboard, when a run is completed. Requirements([R30]-[R31]-[R32])
	\begin{enumerate}
	\item Account service.
	\item Spectator service.
	\end{enumerate}
The Spectator service, furthermore, allows spectator to see the leaderboard on the map during the race. 
	
\item[{[G11]}] Allow organizers (i.e. third parties) to set up a run, by defining its name, its path, date, start time, expiration date, and the minimum number of participants. Requirements([R33]-[R34]-[R45])
	\begin{enumerate}
	\item Account service.
	\item Race service.
	\end{enumerate}
Race service allows third party customer to set up a race and manage it. The service provides an API for posting a race with all the required information.
	
\item[{[G12]}] Allow a third party to access data specified in a request if the user accepts the request or if he accepted one or more requests from the same third party that provided access to the same data. Requirements([R35]-[R36]-[R37]-[R38])
	\begin{enumerate}
	\item Account service.
	\item Individual request service.
	\item Share data service.
	\end{enumerate}
In the \textit{Component view} section the Individual request service has already been discussed and it enables third party customer to upload individual request and monitor their status. Furthermore, it enables a user to accept a request. \\
Share data service is necessary in order to access the data.	
	
\item[{[G13]}] Allow a third party to access statistical and anonymized data if and only if the number of individual involved is greater than 1000. This is satisfied after the request is approved. Requirements([R39]-[R40]-[R41]-[R42])
	\begin{enumerate}
	\item Account service.
	\item Group request service.
	\item Share data service. 
	\end{enumerate}
The Group request service collaborates with share data service, which controls the number of users involved. If it is greater than 1000 the request is accepted and the data is provided to the third party customer, with specific filters required in the request message. 
The access to the data is provided with Share data service.
	
\item[{[G14]}] Allow a third party to subscribe to non-existing data. They will have access to them, after the data is generated. Requirements([R43]-[R44])
	\begin{enumerate}
	\item Account service.
	\item Individual request service.
	\item Group request service.
	\item Share data service.
	\end{enumerate}
	
Each request service allows to subscribe to non-existing data and then after the data is generated, the status of the request will be changed
and the access to the data will be provided.
\end{itemize}
\subsection{Non-functional requirements}
Other requirements like performance, reliability, availability, security, maintainability and portability are now discussed in particular. \\
\subsubsection{Performance}
The system must be able to manage huge number of requests, as written also in the RASD, and for achieving this performance requirement TrackMe
uses a cloud database. 
Furthermore, since the Spectator service and the Share data service requires a elevated throughput, these services are deployed in acluster.
This, also, enhances scalability.
\subsubsection{Reliability}
The application has to be reliable as possible. To reach this requirement the application services are duplicated on several machines and all test are tested very carefully to reduce errors and bugs as much as possible. 
\subsubsection{Availability}
Note that the data are replicated among the services, that increases reliability and performance, but it introduces consistency problems
(this problems and their solution are well discussed before in the \textit{Eventually consistency and compensation } section).
For all these things, the reliability of the application is very high.
\subsubsection{Security}
Also the security and the privacy of data are requirements very
important: each transactions between the user hardware device and the user application, and all temporary data store into the mobile device
are encrypted with AES-128. Moreover, each exchange of data between the user application and the TrackMe server use HTTPS protocol for a high
security communication: indeed, microservices expose REST API, which introduces also other benefits likes separation of concerns between
client and server, reliability and scalability. 
Furthermore, the use of the API Gateway and the access token pattern easiness the management of the security in the system, since it is the
only way to interact with the application and the token certifies the action.
\subsubsection{Maintainability}
The use of external services, framework and the microservices architecture allows to achieve an easy maintainability of the system. 
Also the use of specific patterns helps to achieve this scope.\\
The portability is achieved by developing the application in Android. 
Indeed, after an analysis of the Italy market, has resulted that Android is the most popular mobile system and the frameworks chosen are also
well supported by the current market.

