\subsection{Reference documents}
The DD refers to different documents:
\begin{itemize}
\item Slides about DD of Software Engineering 2
\item old DD documents such as: "the DD to be analyzed", DD of 2015/16 and 2016/17.
\item old Integration and test plan document of 2016/17
\end{itemize}
It, also, refers to many websites:
\begin{itemize}
\item \url{https://microservices.io}
\item \url{http://eventuate.io}
\item \url{https://www.rabbitmq.com}
\item \url{https://www.baeldung.com/transactions-across-microservices}
\item \url{https://logz.io/blog/logging-microservices/}
\item \url{https://dzone.com/articles/microservice-design-patterns}
\item \url{https://dzone.com/articles/communicating-between-microservices}
\item \href{https://medium.com/@walkingtreetech/transaction-management-in-microservices-ab09b0cb803b}{Transaction Management in Microservices}
\item \href{https://dev.to/matteojoliveau/microservices-communications-why-you-should-switch-to-message-queues--48ia}{Microservices communications why you should switch to message queues}
\item \href{https://www.qualogy.com/techblog/it-development-and-operations/microservices-when-not-to-use-them...#}{Microservices when not to use them}
\end{itemize}
The first two websites are very useful to learn the architecture of microservices but not enough. Moreover, there are also interesting videos that explain very well microservices architecture and their pattern:
\begin{itemize}
\item \url{https://www.youtube.com/watch?v=txlSrGVCK18}
\end{itemize}


