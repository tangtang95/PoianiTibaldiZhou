\subsection{Scope}
TrackMe is composed of three main features: Data4Help, AutomatedSOS and Track4Run.
Data4Help is thought as a service that allows to monitor the position and the health status of individuals. 
It provides different types of diagnostic procedures: heartbeat, blood pressure and blood oxygen saturation levels. 
Moreover, Data4Help permits to the users to constantly see their health statuses in order to be conscious of their condition. 
Third party customers can demand for specific data of a particular user, by sending him a request message, or they may ask aggregated data
on a group of user, whose number of members is greater then 1000. User may accept or decline individual request; in this last case they are
also enabled to block a third party customer (i.e. he will be no more able to send request to that user).
TrackMe, also, implements AutomatedSOS service: when the health parameters of a user go below the standards, the system contacts an emergency
room and an ambulance is provided. 
Finally, with Track4Run the organizers can set up a race and all the interested people (i.e. runners) may sign up to the race. 
During the event, the application will automatically monitor the runners, and spectators will be able to follow the race through the system: 
at the end of the race, the organizers have to close the run and the spectators will be able to see a leaderboard. 