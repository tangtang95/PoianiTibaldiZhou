\subsection{Implementation plan}
First of all, since client and server can be developed at the same time: indeed, there is low coupling
between them, and this is also enhanced by the REST API. To give some structure to the chapter, first the
implementation of the server components are discussed, and than the ones regarding the client. \\ 

\par 
A great property of microservices structure is that microservices can be developed independently. 
The implementation of each service will be done by starting from the set up of its DBMS system and then
the main logic of the service. 
Each service can be seen as a completely separate system with its memory stack, its logic and its user
interface. 
At the end, the user interfaces are merged and reachable from a single mobile application. 
The order in which the services are implemented depends on a number of factors, like the complexity of the
service and the dependence of other modules. 
In this sense, the components of the TrackMe server, could be developed in the following order:

\begin{enumerate}
\item Share data service.
This service is one of the more important server side component. It takes the GPS data and the health data
and it shares them through the two interfaces: download and upload data service. 
This is the first module since it is the main core functionality of the whole project. 
\item Individual request service and Group request service.
The request services are the main modules for the interaction with the third party customer, and, also
the one that are more dependent w.r.t. to the core business of Data4Help. For this reason, it follows 
the share data service.
\item Race service.
The integration with maps API and the several features, that it is composed, makes the programming of this
component rather long. Although this fact has already been considered in the design of the system, this
still represents a point of possible problems.
\item Spectator service.
Spectator service is another service rich of feature to implement and its user interface is not so easy.  
This component and the previous one need the configuration of an external service (i.e. Google Maps) 
\item Account service. 
This service is likely the most easy to develop, because it is, somehow, a basic feature for most of the
applications. 
However,the security of the access has not to be underestimated, and its importance is crucial in the
system.
\end{enumerate}   
As one can see from the order, first of all the functions of Data4Help are implemented, since it is
considered to be the most relevant feature of the project. \\ 
Parallel to the service, it is important to implement the message queue, since this will give the 
possibility of starting integration test much earlier. 

After this, for completing the server side, the main parts of communication system must be built, in
particularity the components which are interested are: 
\begin{enumerate}
\item API Gateway: this will be the first one among the remaining server components, since it gives the
possibility to set up the connection with the clients
\item Router and API gateway: this are strongly connected, since they are part of the same architectural
design pattern, so they will be developed simultaneously
\end{enumerate}

\par
As already mentioned, the client side can be developed at the same time of the server side. 
The development of the user application starts with the set up of the communication with the hardware,
like GPS sensor and Call service. 
Then, the user interface of each service, server side, are drawn and developed, so the mock-up of the
application takes shape.