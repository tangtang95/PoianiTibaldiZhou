\subsection{Definitions, Acronyms, Abbreviations}
\subsubsection{Definitions}
\begin{itemize}
\item user: a person who has registered on the system (it is equivalent to subscribed user).
\item athlete: a user who has subscribed to a race and intends to participate as a runner.
\item spectator: a user who wants to watch the runners' position during a race   
\item third party customer: a person or company that wants to access data
\item organizer: a third party customer who can set up a run
\item individual requests: requests that third party customers can send that allow, if permissions are granted, to access the requested data of a specific user
\item aggregated requests: requests that third party customers can send that allow, if permissions are granted, to access aggregated and statistical data on a certain group of user
\item health parameter: a value regarding healthcare status of a specific individual that can be heartbeat, blood pressure or blood oxygen saturation levels 
\item threshold: a critical value regarding health parameters, which is unique for every user
\item starting condition of a run: minimum number of participants
\item race's overlapping path: a path is considered to be overlapping with that of another race, if they share at least a piece of path
\item autonomous call: a call that starts automatically and in which a text-to-speech API is used to communicate with the other end point. 
Furthermore, a voice recognition API is used to retrieve a response. 
For better specifying, in the AutomatedSOS service, the autonomous call contacts the hospital and communicate the health parameters and the user location, and waits the a confirmation (e.g. "OK") from the emergency room operator
\end{itemize}

\subsubsection{Acronyms}
\begin{itemize}
\item RASD - Requirement Analysis and Specification Document
\item NFC - Near Field Communication
\item GPS - Global Positioning System
\item API - Application Program Interface
\item DSS - Data Storage System
\item SSL - Secure Sockets Layer
\item TLS - Transport Layer Security
\item HTTPS - Hypertext Transfer Protocol Secure
\end{itemize}

\subsubsection{Abbreviations}
\begin{itemize}
\item [Gi] - Goal number \textit{i}, where \textit{i} is a number
\item [Ri] - Requirement number \textit{i}, where \textit{i} is a number
\end{itemize}
