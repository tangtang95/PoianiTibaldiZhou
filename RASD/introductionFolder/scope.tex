\subsection{Scope}
As already mentioned, the basic Data4Help service allows to monitor the position and the health status of individuals. When an user registers to the service he accepts the application's contract, that permits the acquirement of user's data from his device. The information, once received, is stored. Data4Help provides different types of diagnostic procedures:  heartbeat, blood pressure and blood oxygen saturation levels.\\ 
Each ten seconds the user's device sends data to TrackMe, that saves them into the system. \\ 
The people, who are probably most interested in this service, are whoever has a particular attention toward the health of himself, their family, or their close friends.
For instance, this application allows parents to monitor their children, when they are unable to stay with them. 
Moreover, Data4Help permits to the users to constantly see their health status in order to be conscious of their condition. Indeed, this will keep patients regularly updated on their progress and will provide proactive measures for a better health control. 
In order to allow a third party customer to see the status of an appointed individual, he needs to send him a request of sharing data by means of the form provided by the system. 
The receiver, obviously, can accept or reject the request according to the sender and the attached reason. 
If the receiver accepts the demand, the requesting customer can see his data, which is related to the date of its generation into the system. Third party customers can also demand for anonymized data. \\ 

\par
Also, with AutomatedSOS service, when their health parameters of an user go below the standard, the system, within 5 seconds, contacts an emergency room.
The ambulance is managed by the owner of the vehicle, and it intervenes within the timing defined by the State.
TrackMe, with this feature, hopes to help hospitals and private specialists to save lives.

\par
For using this service, the organizers need to post a race event on the application, that contains a description and a timetable. 
Then, all the interested people (i.e. runners) must sign up to the race.
Notes that while registering, a runner accepts to share his data during the competition. 
During the event, the application will automatically monitor the runners, and spectators will be able to follow the race on their smartphones. 
Notes that the possibility of downloading data from the Track4Run service is allowed only when the competition is taking place. \\

Summarize, all the features can be classify into two categories: world phenomena and shared phenomena. 
The most relevant world phenomena are:
\begin{enumerate}
\item The management of the ambulance.
\item The possibility of running into a competition.
\item Upload maps. 
\end{enumerate}
While the shared phenomena are:
\begin{enumerate}
\item The sign up into the application.
\item Take data from the sensor.
\item The possibility of seeing the race.
\item Send the request for shared data.
\item The definition of the path for the runner race.
\item Post the competition on the application.
\item The automated emergency call. 
\item The subscription into the race. 
\item The management of the shared data request.
\item The access to the data.
\item Take GPS position.
\end{enumerate}
