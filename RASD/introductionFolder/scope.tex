\subsection{Scope}
As already mentioned, the basic Data4Help service allows to monitor the position and the health status of individuals. When an user registers to the service he accepts the application's contract, that permits the acquirement of user's data from his device. The information, once received, is stored. 
Each ten second the user's device sends data to Data4Help servers that save them into the system; if a device go offline, the last data will be available on the server.\\ 
The people who are probably most interested in this service, is whoever has a particular attention toward his health and toward the health of their family or their close friends (e.g. parents).
For instance, this application allows parents to monitor their children, when they are unable to stay with them. 
Moreover, Data4Help permits to the users to constantly see their health status in order to be conscious of their condition. Indeed, this will keep patients regularly updated on their progress and will provide proactive measures for a better health control. 
In order to allow a third party customer to see the status of an appointed individual, he needs to send him a request of sharing data by means of the form provided by the system. 
Here, he must specify the social security number of the individual and a brief description that motives the request. 
The receiver, obviously, can accept or reject the request according to the sender and the attached reason. 
If the receiver accepts the demand, the requesting customer can see his data, that is related with the date of its registration into the system. \\ 
In addition, the AutomatedSOS service results to be particularly helpful for old people or patients, that, as a matter of fact, are more subjective to health problems: thanks to this instrument they can be assisted in every moment of the day. 
When their health parameters go down under the standard, the system, within 5 seconds, calls the 118 number autonomously, in order to send an ambulance to the user location. 
The ambulance is managed by the owner of the vehicle (hospitals, onlus and privates), and it intervenes within the timing defined by the state.
TrackMe, with this feature, hopes to help hospitals and private specialists to save lives.
Data4Help provides different types of diagnostic procedures: blood pressure monitoring, pulse, and blood oxygen saturation levels.\\ 
The GPS system is exploited also by Track4Run feature, which allows third party organizers to define a certain path for a run, and to manage both participants and spectators during the race.
For using this service, the organizers need to post a race event on the application, that contains a description and a timetable. 
Then, all the interest people (i.e. both runner and spectators) must sign up to the race in their specific section. 
Notes that while registering, a runner accepts to share his data during the competition. 
During the event, the application will automatically monitor the runners, and spectators will be able to follow the race on their smartphones.
The organizers, in addition to the runner position, can also access to the health status of athletes, in order to intervene in case of illness. 
Note that the possibility of downloading data from the Track4Run service is allowed only when the competing is taking place. \\
\par
The TrackMe company business concerns the sale of anonymous statistic data to companies, which can request both the health statuses and the positions of specific groups of people (e.g. people over 40). 
The policy implemented by TrackMe, prevents third parties from finding real owners of data. 
Indeed, a request from a company can be accepted only if the group of people involved contains at least 1000 individuals.