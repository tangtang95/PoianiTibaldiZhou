\subsection{Scope}
As already mentioned, the basic Data4Help service allows to monitor the position and the health status of individuals. When a user registers to the service he accepts the application's contract, that permits the acquirement of his data from his device. The information, once received, is stored. Data4Help provides different types of diagnostic procedures:  heartbeat, blood pressure and blood oxygen saturation levels.\\ 
Periodically, the user's device sends data to TrackMe, that saves them into the system. \\ 
The people, who are probably most interested in this service, are whoever has a particular attention toward the health of himself, their family, or their close friends.
For instance, this application allows parents to monitor their children, when they are unable to stay with them. 
Moreover, Data4Help permits to the users to constantly see their health status in order to be conscious of their condition. Indeed, this will keep patients regularly updated on their progress and will provide proactive measures for a better health control. 
In order to allow a third party customer to see the status of an appointed individual, he needs to send him a request of sharing data by means of the form provided by the system. 
The receiver, obviously, can accept or reject the request according to the sender and the attached reason. 
If the receiver accepts the demand, the requesting customer can see his data, which is related to the date of its generation into the system. Third party customers can also demand for anonymized data. \\ 
Notes, also, that a request can expire: the user has to accept (if he wants to) before the expiration date.

\par
Also, with AutomatedSOS service, when the health parameters of a user go below the standards, the system, within 5 seconds, contacts an emergency room.
The ambulance is managed by the owner of the vehicle, and it intervenes within the timing defined by the State.
TrackMe, with this feature, hopes to help hospitals and private specialists to save lives.

\par
For using the Track4Run service, the organizers need to post a race event on the application, that contains a description and a timetable. 
Then, all the interested people (i.e. runners) must sign up to the race.
Notes that while performing the sign up to TrackMe, a runner accepts to share his data during the competition. 
During the event, the application will automatically monitor the runners, and spectators will be able to follow the race through the system. 
Notes that the possibility of observing the runners' position is allowed only when the competition is taking place. \\

Summarizing, all the features can be classified into two categories: world phenomena and shared phenomena. 
The most relevant non-shared world phenomena are:
\begin{enumerate}
\item The dispatch of an ambulance
\item Athlete is running 
\item A user is ill
\item Organizers set up a run 
\item Athlete shows up in a competition
\item Emergency room operator provides help
\item A user is hospitalized
\item An emergency call is redirected to the nearest hospital 
\item A race terminates 
\end{enumerate}
While the shared phenomena are:
\begin{enumerate}
\item Sign up and log in into the application
\item User watches a competition
\item Third party customers send requests to access data
\item Organizers post the competition
\item The automated emergency call
\item Athlete subscribes / unsubscribes to a race
\item Athlete participates in a run
\item User accepts / declines a request
\item User blocks third party customer 
\item User's position is updated
\item User's health status is updated 
\item Third party customer access requested data
\item A competition is canceled 
\item Organizer signals that a race is closed (i.e. terminated)
\end{enumerate}
