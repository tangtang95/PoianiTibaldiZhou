\subsection{Purpose}
\par
The TrackMe system is designed as a software application for monitoring the position and the health of the user. \\
The application provides three services that are described below. \\
The first one, Data4Help, is thought for all people that want to keep under control their health during the day or who want to always know the position and health status of a particular person in the world. Indeed, with this service a third party user can send a request to access data of some specific user, by means of his social security number: if the receiver agrees, it is possible to see the data regarding the request. The service supports the registration of individual who, by signing in, agrees that the company TrackMe acquires their data, which will be used anonymously by third parties for making statistics on groups of people. \\
The second one is called AutomatedSOS service. It is thought for people that have serious health problems and, in case of illnesses (i.e. some parameters observed below the threshold), the system contacts an hospital. \\
Finally, the Track4Run service is also available. 
It is developed for organizers of sport events that want to monitor the runners in a race. The service allows organizers to set up a run, participants to enroll in the run, and spectators to see the position of all runners during the run on a map. \\

\subsubsection{Goals}
The goals can be distinguished into two families: the former regarding the users, and the latter regarding the third part customers.\\
The ones regarding the subscribed users, are the followings:
\begin{enumerate}
\item[{[G1]}] Allow a user to access its own data
\item[{[G2]}] Allow a user to contribute to data sharing by providing information about his location and health status
\item[{[G3]}] Once the health parameters of a user have been observed 
below the threshold for the first time after one hour, an ambulance is sent to the user location. 
\item[{[G4]}] The time experienced between the moment in which the health parameters of a subscribed user are observed below the threshold and the time in which the emergency point is contacted is equal or less than 5 seconds
\item[{[G5]}] Allow a user to participate in a run managed by third parties, as an athlete, if all starting condition are satisfied
\item[{[G6]}] Allow spectators (i.e. user) to see on real-time the "correct" positions of all athletes taking part in a run, with at most 15 meters of radius error
\item[{[G7]}] The maximum time to accept an individual request from any third party is 30 days; after that, the request will expire
\item[{[G8]}] Allow a user to accept or refuse a request from third parties
\item[{[G9]}] Allow a user to block requests made by a specific third party and all the pending requests will be refused. This action is possible only when the user has already refused one request made
by the customer that he is intending to block.
\item[{[G10]}] Allow spectators and runners to see the leaderboard, when a run is completed
\end{enumerate}
The goals of the project, regarding the third part customers, are the followings:
\begin{enumerate}
\item[{[G11]}] Allow organizers (i.e. third parties) to set up a run, by defining its name, its path, date, start time, expiration date, and the minimum number of participants
\item[{[G12]}] Allow a third party to access data specified in a request if the user accepts the request or if he accepted one or more requests from the same third party that provided access to the same data 
\item[{[G13]}] Allow a third party to access statistical and anonymized data if and only if the number of individual involved is greater than 1000. This is satisfied after the request is approved 
\item[{[G14]}] Allow a third party to subscribe to non-existing data. They will have access to them, after the data is generated. 
\end{enumerate}