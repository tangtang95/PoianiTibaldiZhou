\subsection{Purpose}
\par
The TrackMe system is designed as a distributed software application which demands the usage of a device that dispose of a NFC sensor and a GPS system inside for monitoring the position and the health of the owner. \\
The application provides three services that are described below. \\
The first one, Data4Help, is thought for all people that want to keep under control their health during the day or who want to always know the position and the status of health status of a particular person in the world. Indeed, with this system a third party user can send a request to access data of some specific user, by means of his social security number: if the receiver agrees, it is possible to see in real time the last information registered about that person. The service supports the registration of individual who, by signing in, agrees that the company TrackMe acquires their data, which will be used anonymously by third parties for making statistics on groups of people. \\
The second one is called AutomatedSOS service. It is thought for people that have serious health problems and, in case of illnesses (i.e. some parameters observed below the threshold), the system contacts an hospital. \\
Finally, the Track4Run service is also available. 
It is developed for organizers of sport events that want to monitor the runners in a race. The service allows organizers to define the path of a run, participants to enroll in the run, and spectators to see the position of all runners during the run on a map. \\

\subsubsection{Goals}
The goals can be distinguished into two families: the former regarding the users, and the latter regarding the third part customers.\\
The ones regarding the subscribed users, are the followings:
\begin{enumerate}
\item[{[G1]}] Allow a user to access its own data
\item[{[G2]}] Allow a user to contribute to data sharing by providing information about his location and health status
\item[{[G3]}] Once the health parameters of a user have been observed 
below the threshold, an ambulance is sent to the user location 
\item[{[G4]}] The time experienced between the moment in which the health parameters of a subscribed user are observed below the threshold and the time in which the emergency point is contacted is equal or less than 5 seconds
\item[{[G5}] Allow a user to participate in a run managed by third parties, as an athlete
\item[{[G6]}] Allow spectators (i.e. user) to see on real-time the "correct" positions of all athletes taking part in a run, with at most 15 meters of radius error
\item[{[G7]}] Allow organizers (i.e. third parties) to set up a run, by defining its path, date and start time
\item[{[G8]}] The maximum time to accept an individual request from any third party is 30 days; after that, the request will expire
\item[{[G9]}] Allow a user to block requests made by a specific third party
\item[{[G10]}] In case it is impossible to start a run, the participants are notified
\end{enumerate}
The goals of the project, regarding the third part customers, are the followings:
\begin{enumerate}
\item[{[G11]}] Allow a third party to access the data on a certain individual if and only if he accepts. This is satisfied as soon as the request is approved
\item[{[G12]}] Allow a third party to access statistical and anonymized data only on groups of individual greater than 1000. This is satisfied as soon as the request is approved
\item[{[G13]}] Allow a third party to subscribe to non-existing data. They will have access to them, as soon as the data is generated. 
\end{enumerate}