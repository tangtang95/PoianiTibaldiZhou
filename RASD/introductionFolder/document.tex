\subsection{Document structure}
The document structure follows what is defined on the "IEEE standard on requirement engineering":
\begin{itemize}
\item The first section introduces the project, its goals and its world and shared phenomena. Then, it specifies special keywords to try to resolve some ambiguity that might appear to the reader.
\item The second section of this document tries to specify better the system to be implemented. First, it states better what the product to be is and what its core functions are. Furthermore, in this section, it is defined who this application is designed for and the domain assumptions regarding the world environment.
\item The third section is more about requirements. But before specifying which requirement needs to be implemented, it states what are the interfaces of the application (i.e. software, hardware, communication) are; then for a better understanding of the use case, it shows a list of possible scenarios and their corresponding use case. And only after these, requirements are listed (functional and non-functional requirements).
\item The fourth section is all about Alloy. It is listed all the code written to model the system application and maybe some comments about it to define it better.
\item The fifth section shows more or less the effort spent by each author of this document.
\end{itemize}