\subsection{Data for help tests}
\subsubsection{G1}
The uses cases that concerns G1 are: UC1, UC2, UC3, UC4, UC5, UC6, UC7 and UC9.  \\

\begin{itemize}
\item For what concerns UC1, the successful case has been tested (no exception is present), by using some requirements that involves the
third party registration and the possibility to send individual request. In particular, after having created a new third party customer,
and a new user; a request from that customer to the user is made. The user retrieves the list of pending requests (that contains only that
element) and approves and rejects it.

\item For testing UC2 a great number of third party customers has been created and registered: they all send a request to the same user.
The user retrieves all the pending requests and a check on the number of pending request: the behaviour is correct.
The exceptional case, requires the use of the front end: in particular, it is said that if no pending requests are available, then a message
should been shown to the user: however, no message is displayed.

\item UC3 TODO

\item UC4 TODO

\item The test of UC5 has been performed togheter with UC1: after that the request is approved, we have checked that in the subscriptions
of the third party there exists one on a SSN that is equal to the user that accepted the request.

\item UC6 involves the sign up of user and individual requests. The successful case has been tested and also the cases in which SSN and email
are already present in the system. \\
The result is the following: userId and accessToken returned by the server are correct. In particular, they have been used to check
if the user was registered by performing an http request to retrieve the user's information.  \\
The exception about checking if fields are empty or not has been skipped since this is something that is implemented correctly in the front
end.

\item UC7 consists in logging users and third parties into the system. The normal event flow and the cases of wrong credential has been
tested.
The same check adopted in UC6 has been used here. \\ 
The exception that involves missing password or email, has been skipped since this is something that is implemented correctly in the front 
end.

\item UC9 TODO

\end{itemize}


\subsubsection{G2}
The uses cases that regards G2 are: UC6, UC7, UC8, UC9, UC32 and UC33.  \\


\begin{itemize}
\item The first two use cases (i.e. UC6 and UC7) were already presented above, while discussing G1.

\item To test UC8, basically, the idea is searching for the user SSN. 
If it is found, it returns a bad request (i.e. this seems strange, but it is the correct chain-flow) with a specific message "Should send a
request to the individual to access his data". This happens because the third party customer is not subscribed to the data of that user, and
therefore a request has to be sent.
Thus, after that, a request is sent: to check if things work properly, we have to be sure that the user has received a new pending request. 
The result is correct.  \\
The exceptions have been tested successfully. 

\item UC9 TODO

\item UC32 and UC33 have not been implemented by the other team. TODO COMMENT ON WHAT YOU GET

\end{itemize}

\subsubsection{G3}
The uses cases that involves G3 are: UC6, UC7, UC11 and UC12.  \\

\begin{itemize}
\item 
As already mentioned, the first two use cases are just tested above during the test of G1.

\item 
During the test of UC11 is impossible to understand when there is an error message. 
In fact by doing the manual test, even 3 blocks of anonymized data are returned and no error message is shown. 
This is in conflict with what is expressed in the description of the use case of the RASD document. TODO \\

TODO CHECK WHAT IS WRITTEN HERE: TOTALLY NO SENSE, USE SOME CONTEXT
For checking the creation of new data into the system just leave the server up. 
The more time the server is up, the more
data there are (i.e. every minute new data arrives), therefore we just check if the response is OK.

\item Since UC12 seems to equivalent to UC11, therefore the results are already discussed in the precedent point

\end{itemize}

\subsubsection{G4}
The uses cases that concerns G4 are: UC6, UC7, UC9, UC10, UC11, UC12, UC32, UC33 and UC34. \\

\begin{itemize}
\item 
UC6, UC7, UC9, UC11, UC12, UC33, UC32 are already discussed above

\item 
UC10 consists in the subscription of a third party customer to new user data. \\
In the website, the subscribe is done when a checkbox is crossed during the search of bulk data. 
While testing with JMeter, the search has been skipped and only an HTTP request to directly subscribe has been done.
To check if things work properly, the subscriptions of that third party are retrieved: since the subscription id is present, the
result is correct.

\item UC31 has not been implemented by the other team. TODO DESCRIBE THE BEHAV.

\end{itemize}

\subsubsection{Other tests}
TODO ALL (if you are going to test only with the front end, please, change the description of the structure mentioned above) 