\subsection{Data for help tests}
\subsubsection{G1}
The uses cases that concerns G1 are: UC1, UC2, UC3, UC4, UC5, UC6, UC7 and UC9.  \\

\par 
For what concerns UC1, the successful case has been tested (no exception is present), by using some requirements that involves the
third party registration and the possibility to send individual request. In particular, after having created a new third party customer,
and a new user; a request from that customer to the user is made. The user retrieves the list of pending requests (that contains only that
element) and approves and rejects it.

\par 
UC2 to be completed

\par 
UC3

\par 
UC4

\par 
UC5

\par 
UC6 involves the sign up of user and individual requests. The successful case has been tested and also the cases in which SSN and email
are already present in the system. \\
The result is the following: userId and accessToken returned by the server are correct. In particular, they have been used to check
if the user was registered by performing an http request to retrieve the user's information.  \\
The exception about checking if fields are empty or not has been skipped since this is something that is implemented correctly in the front
end.

\par 
UC7 consists in logging users and third parties into the system. The normal event flow and the cases of wrong credential has been tested.
The same check adopted in UC6 has been used here. \\ 
The exception that involves missing password or email, has been skipped since this is something that is implemented correctly in the front 
end.



\subsubsection{G2}
The uses cases that concerns G2 are: UC6, UC7, UC8, UC9, UC32 and UC33.  \\

The first two use cases are already tested above during the test of G1.

\par
For what concerns UC8 basically the idea is searching for the user SSN. If it is found it returns a bad request (i.i. this seems strange but it is the correct chainflow) with a specific message "Should send a request to the individual to access his data". Then it will send the request: to check if it works in the test and after it checks if the user received has a new pending request.
The result of the test is correct, but the choice to use a bad request message as an answer to a correct operation is a bit strange. Also the exception 2 has been tested and it checks that an error message will be shown.

\par
UC9 Test on this use case is skipped.

\par
Instead UC32 and UC33 have not been implemented by the other team.


\subsubsection{G3}
The uses cases that concerns G3 are: UC6, UC7, UC11 and UC12.  \\
As already mentions the first two use cases are just tested above during the test of G1.

\par
During the test of UC11 is impossible to understand when there is an error message. In fact by doing the manual test, even 3 blocks of anonymized data are returned and no error message is shown. This is in conflict with that is expressed in the description of the use case of the RASD document. For checking the creation of new data into the system just leave the server up. The more time the server is up, the more data there are (i.i. every minute new data arrives), therefore we just check if the response is OK.

\par
Since UC12 seems to equivalent to UC11, therefore results already checked.


\subsubsection{G4}
The uses cases that concerns G4 are: UC6, UC7, UC9, UC10, UC11, UC12, UC32, UC33 and UC34. \\

UC6, UC7, UC9, UC11, UC12, UC33, UC32 already discussed above.

\par 
UC10 consists in the subscription to new user data by third party customer. This case has been tested with the website. The subscribe is done when a checkbox is crossed during the search of the bulk data. Here in this test of jmeter, the HTTP method of post subscriptions has been done without doing the search of bulk data. And then to check if it works, we went to the subscription of the third party and see if there is the one that has been inserted (by checking subscription id).

\par 
UC34 has not been implemented by the other team.

\subsubsection{Other tests}