\section{Acceptance tests}
The approach to acceptance testing, for both data for help and automated SOS, has been
conducted in the following way. 
First of all, it has been decided to test all the possible use cases that are described in the
second version of the requirements and analysis document (of course, only the scenarios that
regards the implemented requirements and goals, mentioned in the implementation and document testing, have been considered). \\
The exceptions of the various use cases have been considered too, since the system should
behave properly, also, in these conditions. \\
Secondly, the website has been tested manually, by exploring the UI and its possibilities. \\

The tests that regards the user cases have been performed with JMeter. The source code is present in the delivery folder. 
Before starting the testing phase it is necessary to know the structure of the JSON files that are sent to the server. To do this we used
one of the JMeter feature: HTTP(S) Test Script Recorder which can be used to record all the requests a web
application is making to the server. As the name suggests, it will capture only the HTTP(s) requests. To make it work, it is necessary to set a Web Proxy on the browser to localhost:port (e.g. 8888 the standard port of the Test Script Recorder) and start the recording. After this it is possible to navigate on the browser and every request will be recorded and added on the "Recorder Controller" (look RecordingTemplate.jmx file)

\par
The cases are presented in the next two subsections, each of them, first exposes the
tests performed on the use cases divided by goal. After that, the manual tests on the UI are presented. \\
Finally, at the end of this chapter, it is possible to find some brief conclusions. 

\subsection{Data for help tests}
\subsubsection{G1}
The uses cases that concerns G1 are: UC1, UC2, UC3, UC4, UC5, UC6, UC7 and UC9.  \\

\begin{itemize}
\item For what concerns UC1, the successful case has been tested (no exception is present), by using some requirements that involves the
third party registration and the possibility to send individual request. In particular, after having created a new third party customer,
and a new user; a request from that customer to the user is made. The user retrieves the list of pending requests (that contains only that
element) and approves and rejects it.

\item For testing UC2 a great number of third party customers has been created and registered: they all send a request to the same user.
The user retrieves all the pending requests and a check on the number of pending request: the behaviour is correct.
The exceptional case, requires the use of the front end: TODO

\item UC3 TODO

\item UC4 TODO

\item The test of UC5 has been performed togheter with UC1: after that the request is approved, we have checked that in the subscriptions
of the third party there exists one on a SSN that is equal to the user that accepted the request.

\item UC6 involves the sign up of user and individual requests. The successful case has been tested and also the cases in which SSN and email
are already present in the system. \\
The result is the following: userId and accessToken returned by the server are correct. In particular, they have been used to check
if the user was registered by performing an http request to retrieve the user's information.  \\
The exception about checking if fields are empty or not has been skipped since this is something that is implemented correctly in the front
end.

\item UC7 consists in logging users and third parties into the system. The normal event flow and the cases of wrong credential has been
tested.
The same check adopted in UC6 has been used here. \\ 
The exception that involves missing password or email, has been skipped since this is something that is implemented correctly in the front 
end.

\item UC9 TODO

\end{itemize}


\subsubsection{G2}
The uses cases that regards G2 are: UC6, UC7, UC8, UC9, UC32 and UC33.  \\


\begin{itemize}
\item The first two use cases (i.e. UC6 and UC7) were already presented above, while discussing G1.

\item To test UC8, basically, the idea is searching for the user SSN. 
If it is found it returns a bad request (i.e. this seems strange, but it is the correct chain-flow) with a specific message "Should send a
request to the individual to access his data". TODO COMMENT MORE ON THIS BEHAV.
After that, a request is sent: to check if things work properly, we have to be sure that the user has received a new pending request. 
The result is correct, but a the status of the HTTP request is 400 (i.e. bad request): in our opinion, this is not very good.  \\
The exceptions have been tested successfully. 
TODO CHECK IF WHAT WRITTEN IS CORRECT

\item UC9 TODO

\item UC32 and UC33 have not been implemented by the other team. TODO COMMENT ON WHAT YOU GET

\end{itemize}

\subsubsection{G3}
The uses cases that involves G3 are: UC6, UC7, UC11 and UC12.  \\

\begin{itemize}
\item 
As already mentioned, the first two use cases are just tested above during the test of G1.

\item 
During the test of UC11 is impossible to understand when there is an error message. 
In fact by doing the manual test, even 3 blocks of anonymized data are returned and no error message is shown. 
This is in conflict with what is expressed in the description of the use case of the RASD document. \\

TODO CHECK WHAT IS WRITTEN HERE: TOTALLY NO SENSE, USE SOME CONTEXT
For checking the creation of new data into the system just leave the server up. 
The more time the server is up, the more
data there are (i.e. every minute new data arrives), therefore we just check if the response is OK.

\item Since UC12 seems to equivalent to UC11, therefore the results are already discussed in the precedent point

\end{itemize}

\subsubsection{G4}
The uses cases that concerns G4 are: UC6, UC7, UC9, UC10, UC11, UC12, UC32, UC33 and UC34. \\

\begin{itemize}
\item 
UC6, UC7, UC9, UC11, UC12, UC33, UC32 are already discussed above

\item 
UC10 consists in the subscription of a third party customer to new user data. \\
In the website, the subscribe is done when a checkbox is crossed during the search of bulk data. 
While testing with JMeter, the search has been skipped and only an HTTP request to directly subscribe has been done.
To check if things work properly, the subscriptions of that third party are retrieved: since the subscription id is present, the
result is correct.

\item UC31 has not been implemented by the other team. TODO DESCRIBE THE BEHAV.

\end{itemize}

\subsubsection{Other tests}
TODO ALL (if you are going to test only with the front end, please, change the description of the structure mentioned above) 
\subsection{Automated SOS tests}
As indicated by the other team, the Automated SOS service is based on: a schedule job executed every 24 hours to send request regarding the usage of the ASOS service to the elderly people (over 60 years old). To avoid waiting so much time, we did a little modification on the source code of the project. One class called "src/data4help/src/main/java/avila/schiatti/virdi/Main.java" has to be modified:
\begin{itemize}
\item Substitute the code
\begin{verbatim}
	ASOSRequestScheduler.create()
\end{verbatim}
with
\begin{verbatim}
	ASOSRequestScheduler.create().setPeriod(10).setTimeUnit(TimeUnit.SECONDS)
\end{verbatim}
\item Add the import of the class TimeUnit on the top of the class
\begin{verbatim}
	 import java.util.concurrent.TimeUnit;
\end{verbatim}
\end{itemize}
After this modification, we still could not test the functionality of ASOS. Then after the other team fixed the bug, we tested it and we saw that the new user received a pending requests from Automated SOS. This test regarding UC13 has been done manually due to some unknown problem in JMeter (no new requests were found). 
\begin{itemize}
\item UC13 regards the creation of ASOS Request to elderly people
\end{itemize}

\begin{figure}[H]
\includegraphics[width=\linewidth]{images/pendingASOS}
\caption{ UI Received ASOS requests }
\label{fig:asosrequest}
\end{figure}

The other UC14, UC15 and UC16, regarding receiving health data, sending health data and contacting emergency point, cannot be checked since the send of data cannot be done manually.

\subsection{Conclusion}
The prototype has been developed until a good point, but it still contains some bugs. In particular, many of them regards the user interface
and the website. \\
For what concerns the single goals, we can say the following:

\begin{itemize}
\item
It is possible to state that G1 has been reached. However there are some problems with the notification of third party when individual
requests are approved or rejected.

\item
A same reasoning can be applied also to G2. Indeed, UC32 and UC33 present problems with the notifications

\item
G3 has not been fully reached. To be precise, there is the problem with the constraint on 1000 individuals while accessing bulk data

\item 
G4 is achieved correctly

\item 
For what concerns G5, it has been possible to test graphically only the fact that, after having registered a user older than 60 years old, 
and when 10 seconds are elapsed, a request arrives from ASOS, that is some kind of special third party customer. \\
As mentioned in the section regarding the test of G5, the other functionalities could not be tested.\\

\par
In conclusion, the prototype is not ready to be deployed on the market due to previous problems.

\end{itemize}


