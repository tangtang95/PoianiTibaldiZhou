\documentclass[a4paper]{article}
\usepackage[T1]{fontenc}
\usepackage[utf8]{inputenc}
\usepackage[english]{babel}
\usepackage{graphicx}
\usepackage{float}
\usepackage{hyperref}
\usepackage{makeidx}
\usepackage{caption}
\usepackage{tabularx}
\usepackage[table, xcdraw, dvipsnames]{xcolor}

\graphicspath{ {images/} }
% !TeX spellcheck = en_US
\makeindex

\begin{document}
\title{TrackMe: Implementation document \\Software Engineer 2 - 2018/2019}
\author{
        Riccardo Poiani, Mattia Tibaldi, Tang-Tang Zhou \\
        Politecnico di Milano\\\\ 
        Version 1.0 \\
        \thanks{
		Link to source code:
        \url{https://github.com/tangtang95/PoianiTibaldiZhou/tree/master/Implementation}}
        \thanks{
	    Link to what has to be installed:
	    \url{https://github.com/tangtang95/PoianiTibaldiZhou/releases} and the software specified in the installation instruction.}
}
\maketitle
\newpage
\tableofcontents
\newpage

\section{Introduction and group reference}
In this document, the acceptance testing performed on the TrackMe project of the following people is presented and explained. \\ \\
Name of the authors: Avila, Schiatti and Virdi.  \\
Link to the repository: \url{https://github.com/lauricdd/AvilaSchiattiVirdi}. \\
Document considered for the acceptance testing:
\begin{itemize}
\item Design document 2, present in the delivery folder
\item Requirements analysis and specification document 2, present in the delivery folder
\item Implementation and testing document, present in the delivery folder
\end{itemize}



% !TeX spellcheck = en_US
\section{Requirements and functions implemented}
As already mentioned, in this section it is possible to find information regarding the requirements and the functions that are actually implemented with some motivations.

\subsection{Core requirements and functions}
All the core requirements from R1 to R9 have been implemented, since, as the name suggests, they are fundamental. 

\subsection{Data4Help requirements and functions}
All the requirements related to G1, G2, G8, G9, G12 and G13 have been implemented, basically because they 
are considered as the main features and components of Data4Help. 
This is true for all the mentioned goals except for what concerns G9, that is the blocking of third party
customers, that can be considered as an additional functions, but can always be useful for engaging future users.
The excluded ones are the following
\begin{itemize}
\item G7, that is the management of elapsing requests 
\item G14, that is the subscription to non existing data
\end{itemize}
The motivation that stands behind this choice is basically a constraint on development time and the
fact that these were considered more as a nice feature to have, and not something really critical. 
Indeed, it is still possible to have a good prototype of the service, even without these features.
It should be noted, however, that they do not require big efforts and can be easily integrated into the
project in a second time: in particular, G7 can be considered as a periodical task to be scheduled that
checks and manages the time stamp of the pending requests. 
For what concerns G14, instead, the discussion is a bit more complex, but it basically consists in introducing a new status for the
requests and a task that operates with the requests that are in such status and that performs some checks w.r.t. to the provided dates 
(that are the starting date and the ending date of a request: this will be clear after inspecting the source code of the entities that regard the
requests). \\
A final note on the requirements of Data4Help is the following and it regards the type of aggregated data that third parties can request.  
All the filters mentioned in the design document have been developed, from a server-side point of view. 
For clarifying this statement, that may sound obscure, consider that every available filter, except the one that regards GPS position data, is
only related to some plain input that a third party customer inserts in the application and that is sent to the system.
However, for filtering on GPS position a third party should define the interested region by specifying the coordinates that specify the bounds
of the interested region. 
Since, of course, this is totally no user friendly, the application could provide a list of possible cities and region and automatically
translates it. Of course, this feature could also be deployed directly on the server. \\
Nevertheless, up to now, what is present is the possibility of inserting group requests specifying the GPS filters, but this has not been 
implemented in the mobile application for the discussed reason, and, because, at this stage, it is considered sufficient to have all the 
other filters. 

\subsection{AutomatedSOS}
The goals, and the related requirements, that involves AutomatedSOS are G3 and G4. 
However, in this case, not many of the requirements have been implemented, also due to some external limitations: indeed, considering
the Android system and hardware, it is not possible to just intercept the phone call and access the microphone (for automating the phone call)
and the speakers (to retrieve the response). 
Therefore, in order to fully develop the requirements it would be necessary to use VoIP API, that requires some sort of payments.
However, some requirements such as R12, R13, R15 and R17 have been implemented anyway, with the difference that the VoIP calls are mocked
with normal phone calls and no automation is present for interacting with the emergency room operator. 
This has been considered enough for a viable product also because it has been chosen to give more importance to the core business of the 
company, that is considered to be Data4Help. 
Another few words should be spent on R17, that is "during phone calls, the GPS is set on high precision": this is automatically performed
by the Android system when calling emergency services \cite{androidELS}. \\
Finally, for what concerns the part regarding the 5 seconds the following shrewdness have been adopted: the maxitum timeout for retrieving 
the position takes in the worst case 1 second. 
After that, the more critical part is parsing a JSON that contains the emergency numbers of the various countries, but this weights only
120kB. 
It is thought that this should be fast enough to satisfies the requirements of the five seconds. 
However, the process speed of code is a problem always related with the performance of the user device and the consumed resources of other 
applications, therefore being 100\% sure that the task is accomplished within that time is impossible. Nevertheless, some tests are perfomed
and the average speed satisfies this requirement.

\subsection{Non functional requirements}
For what concerns the non functional requirements, as mentioned in the Design document, a microservice architecture has been developed. 
In particular, the main implemented features of the architecture are the following: 
\begin{itemize}
\item Communication between microservices, since this is crucial in order to have a viable product. Indeed, without this, almost no
requirement could be fully accomplished
\item API Gateway that performs also authentication and authorization: this is also a core feature, since it is the entry point for accessing
all the REST API that the various microservices provide
\item Service registry, because otherwise, one have to set up all the network communication in a more static way, and also the management
of forwarding requests from the gateway would have been more complex 
\item A basic load balancer, called Ribbon, is already present in the API gateway that have been used (i.e. see Zuul in the adopted developed framework 
chapter)
\end{itemize}

\par Some basic security feature have been developed also: indeed, all the passwords are stored in the database with bcrypt and the only type
of communication allowed between clients and the API gateway, is HTTPS. In order to do this, a custom SSL certificate has been
generated using the java keytool \cite{httpstool}.
\\
The authentication have been implemented by means of an UUID random token and the APIs have been secured in such a way that a certain client
can access only the method that he should access: for example, a user cannot access methods that regards a third party customer, but, also
a user cannot access API that regards another user (e.g. a user can access only his pending request, and not the pending requests of any
user) \\
As one can claim, using UUID random tokens is for sure not the best way of achieving authentication, however, the code has been designed
in such a way that is easily possible to upgrade this to the usage of JWT just implementing a single interface. This choice
has been done in order to simplify a bit the security issue and to focus more on the business core of the project. 

\subsection{Other comments}
The database regarding the collected health and position data, has not been deployed on the cloud at this
early stage, because the integration was not considered necessary. Indeed, it is more something that regards the deploy, rather than the 
implementation itself. \\
The same reasoning has been applied also for the deployment of JARs in docker containers. \\
Some checks, from a server side point of view, have not been implemented: for example, when registering a third party customer have to
provide an email: it is not checked that the string constitutes a valid email. Things like this have been considered not crucial 
for a prototype, compared, for example, with the set up of the distributed system or to the business function.
Nevertheless, some controls have been made in the Android application. \\ As regards the andorid application, additional features such as the setting screen and the schematic of the user profile have not been implemented due to lack of time. Furthermore, the legal notices in the use of the application have not been explicated in the appropriate popUp.

\section{Adopted development framework}
Most of the choices that regards the frameworks were already briefly introduced and motivated in the
section 2.7 of the design document (that is, other design decision). 
However, here frameworks and other APIs are recapped:

\begin{itemize}

\item Spring\cite{spring} Boot has been adopted since it fits well for microservices and the set up of pattern components
of the architecture (e.g. service registry) can be easily integrated with the usage of Spring Cloud.
The main drawback of this choice is that spring boot does not offer much control to the developer: this "of
course" limit the development time, but when something goes wrong, it may take time to fix the issue.

\item Spring Security has been used to develop the authentication and the access control of the application.
It is basically the de-facto standard for securing Spring-based application. 

\item Spring Cloud Netflix has been used to integrate and set up the API Gateway and the service registry.
In particular, Zuul has been set up as the API gateway integrated with ribbon and the load balancer, which is already incorporated in Zuul and Eureka as the service registry. 
These were of a great help because it is only necessary to find the right configuration settings and then everything works as expected.

\item RabbitMQ\cite{rabbitMQ} is used to set up the communications among microservices. This was chosen because of its properties: durable, 
open-source, asynchrony and mostly for the reliability. With the help of RabbitMQ it has been achieved a communication 
between microservices which is asynchronous, durable and reliable. Even if RabbitMQ crashes, once it restarts the queues 
and messages are saved and, therefore, nothing is lost. Moreover, in the remote case where a microservice cannot handle 
the message due to an unavailability of other machines such as databases, the message is re-pushed into the same queue and it 
will be re-pulled until the operation has finished correctly. In conclusion, RabbitMQ is an essential piece to achieve the property 
of "Eventually Consistency".

\item Android: the mobile application has been developed for android, targets SDK version 27 and min SDK version 24. Here, Butter Knife has been used in order to easily bind the layout with the activities. It also enables to automatically configure listener to onClick methods.

\end{itemize}

Moreover, not already cited APIs has been adopted, and, therefore, are listed: 

\begin{itemize}
\item Project Lombok, that is a java library that automatically, by use of annotators, creates certain code in order to reduce the amount
of boilerplate code that one write 
\item Jayway JsonPath for manipulating JSON 
\item MySQL has been adopted as a DBMS
\item JPA for the management of persistence and object/relation mapping
\item Google guava for the usage of immutable maps
\item Jackson, that is an high performance JSON processor for Java has been used, since it the default library used by Spring boot 
to convert object to json and viceversa. 
This has been very useful in the set up of the controllers: indeed, it was possible to define POJOs as controller attributes, and the conversion between HTTP requests and Java is perfectly handled. 
Jackson has been used also to define views in controller method: this allows to set up that in certain controller methods only certain attributes of a POJO are used. 
For instance one can specify that when the user is accessing its own information, his password is not sent back, but, when registering, of course the password is needed.
Using different views in different methods helps in achieving what mentioned before
\item HATEOS is used to provide hypermedia contents to the clients. This helps the client mobile application in accessing the right methods 
and it makes the APIs RESTful
\item QueryDSL has been adopted to perform the custom query to access the group request data
\item GeoNames has been used in order to find the country codes, if the Android geocoder doesn't work properly
\item JaCoCo has been used to generate reports on the test coverage (they will be shown in the chapter related to testing)
\item JMeter has been used to test the system as a whole: functional testing, load testing, and a simulation of a real load testing where there is some kind of think time for each user thread
\item Room, a library about SQLite DB has been used to save data regarding health, position and emergency calls.
\item Android bluetooth to communicate with the device collecting health data
\item Android espresso to test mainly the correct execution of the emergency calls (e.g. if a real call is performed or not)
\item support library package: adds support for the Action Bar user interface design pattern and it includes support for material design user interface implementations. 
\end{itemize}


\section{Structure of the source code}

\subsection{Microservices}
Here the code regarding the microservices architecture is explained. \\
The source code that meets the requirements mentioned above has been organized in the following way: for each microservices a project
has been set up. 
Indeed, when dealing with this type of architecture, one should think of a microservice
as a project that should me as much independent as possible from the others: this is the reason that stands behind the choice that has been
made. 
Of course, in this way, it is possible to easily generate the single jars that will be deployed, when necessary, with, as mentioned
in the design document, dockers. Therefore, the following projects are present: API gateway, service registry, group individual request
service, individual request service and share data service. 
As one may notice, the one containing the set up of the API gateway also accesses all the information related with the accounts, and, therefore, authentication and authorization functions are coded here.
In the following sections the structure of the single projects are analyzed. 

\subsubsection{API Gateway}

\subsubsection{Service registry}

\subsubsection{Individual request service}

\subsubsection{Group request service}

\subsubsection{Share data service}

\subsection{Mobile code}

\section{Acceptance tests}
The approach to acceptance testing, for both data for help and automated SOS, has been
conducted in the following way. 
First of all, it has been decided to test all the possible use cases that are described in the
second version of the requirements and analysis document (of course, only the scenarios that
regards the implemented requirements and goals, mentioned in the implementation and document testing, have been considered). \\
The exceptions of the various use cases have been considered too, since the system should
behave properly, also, in these conditions. \\
Secondly, the website has been tested manually, by exploring the UI and its possibilities. \\

The tests that regards the user cases have been performed with JMeter. The source code is present in the delivery folder. 
Before starting the testing phase it is necessary to know the structure of the json files that are sent to the server. To do this you can use
one of the JMeter features and create a Test Script Recorder. HTTP(S) Test Script Recorder can be used to record all the requests which web
application is making to server. As the name suggests, it will capture only the HTTP(s) requests. It can also use grouping option in this
recorder to organize the requests in such a way that similar requests are stored together.(i.e. in the source code, in Thread group section
all the message are stored) Add URL patterns in include and exclude list as per your requirement. In such way is possible to know how are
structured the request and then is possible to create some one of new for testing.

\par
The cases are presented in the next two subsections, each of which first exposes the
tests performed on the use cases divided by goal. After that, the manual tests on the UI are presented. \\
Finally, at the end of this chapter, it is possible to find some brief conclusions. 

\subsection{Data for help tests}
\subsubsection{G1}
The uses cases that concerns G1 are: UC1, UC2, UC3, UC4, UC5, UC6, UC7 and UC9.  \\

\begin{itemize}
\item For what concerns UC1, the successful case has been tested (no exception is present), by using some requirements that involves the
third party registration and the possibility to send individual request. In particular, after having created a new third party customer,
and a new user; a request from that customer to the user is made. The user retrieves the list of pending requests (that contains only that
element) and approves and rejects it.

\item For testing UC2 a great number of third party customers has been created and registered: they all send a request to the same user.
The user retrieves all the pending requests and a check on the number of pending request: the behaviour is correct.
The exceptional case, requires the use of the front end: TODO

\item UC3 TODO

\item UC4 TODO

\item The test of UC5 has been performed togheter with UC1: after that the request is approved, we have checked that in the subscriptions
of the third party there exists one on a SSN that is equal to the user that accepted the request.

\item UC6 involves the sign up of user and individual requests. The successful case has been tested and also the cases in which SSN and email
are already present in the system. \\
The result is the following: userId and accessToken returned by the server are correct. In particular, they have been used to check
if the user was registered by performing an http request to retrieve the user's information.  \\
The exception about checking if fields are empty or not has been skipped since this is something that is implemented correctly in the front
end.

\item UC7 consists in logging users and third parties into the system. The normal event flow and the cases of wrong credential has been
tested.
The same check adopted in UC6 has been used here. \\ 
The exception that involves missing password or email, has been skipped since this is something that is implemented correctly in the front 
end.

\item UC9 TODO

\end{itemize}


\subsubsection{G2}
The uses cases that regards G2 are: UC6, UC7, UC8, UC9, UC32 and UC33.  \\


\begin{itemize}
\item The first two use cases (i.e. UC6 and UC7) were already presented above, while discussing G1.

\item To test UC8, basically, the idea is searching for the user SSN. 
If it is found it returns a bad request (i.e. this seems strange, but it is the correct chain-flow) with a specific message "Should send a
request to the individual to access his data". TODO COMMENT MORE ON THIS BEHAV.
After that, a request is sent: to check if things work properly, we have to be sure that the user has received a new pending request. 
The result is correct, but a the status of the HTTP request is 400 (i.e. bad request): in our opinion, this is not very good.  \\
The exceptions have been tested successfully. 
TODO CHECK IF WHAT WRITTEN IS CORRECT

\item UC9 TODO

\item UC32 and UC33 have not been implemented by the other team. TODO COMMENT ON WHAT YOU GET

\end{itemize}

\subsubsection{G3}
The uses cases that involves G3 are: UC6, UC7, UC11 and UC12.  \\

\begin{itemize}
\item 
As already mentioned, the first two use cases are just tested above during the test of G1.

\item 
During the test of UC11 is impossible to understand when there is an error message. 
In fact by doing the manual test, even 3 blocks of anonymized data are returned and no error message is shown. 
This is in conflict with what is expressed in the description of the use case of the RASD document. \\

TODO CHECK WHAT IS WRITTEN HERE: TOTALLY NO SENSE, USE SOME CONTEXT
For checking the creation of new data into the system just leave the server up. 
The more time the server is up, the more
data there are (i.e. every minute new data arrives), therefore we just check if the response is OK.

\item Since UC12 seems to equivalent to UC11, therefore the results are already discussed in the precedent point

\end{itemize}

\subsubsection{G4}
The uses cases that concerns G4 are: UC6, UC7, UC9, UC10, UC11, UC12, UC32, UC33 and UC34. \\

\begin{itemize}
\item 
UC6, UC7, UC9, UC11, UC12, UC33, UC32 are already discussed above

\item 
UC10 consists in the subscription of a third party customer to new user data. \\
In the website, the subscribe is done when a checkbox is crossed during the search of bulk data. 
While testing with JMeter, the search has been skipped and only an HTTP request to directly subscribe has been done.
To check if things work properly, the subscriptions of that third party are retrieved: since the subscription id is present, the
result is correct.

\item UC31 has not been implemented by the other team. TODO DESCRIBE THE BEHAV.

\end{itemize}

\subsubsection{Other tests}
TODO ALL (if you are going to test only with the front end, please, change the description of the structure mentioned above) 
\subsection{Automated SOS tests}
As indicated by the other team, the Automated SOS service is based on: a schedule job executed every 24 hours to send request regarding the usage of the ASOS service to the elderly people (over 60 years old). To avoid waiting so much time, we did a little modification on the source code of the project. One class called "src/data4help/src/main/java/avila/schiatti/virdi/Main.java" has to be modified:
\begin{itemize}
\item Substitute the code
\begin{verbatim}
	ASOSRequestScheduler.create()
\end{verbatim}
with
\begin{verbatim}
	ASOSRequestScheduler.create().setPeriod(10).setTimeUnit(TimeUnit.SECONDS)
\end{verbatim}
\item Add the import of the class TimeUnit on the top of the class
\begin{verbatim}
	 import java.util.concurrent.TimeUnit;
\end{verbatim}
\end{itemize}
After this modification, we still could not test the functionality of ASOS. Then after the other team fixed the bug, we tested it and we saw that the new user received a pending requests from Automated SOS. This test regarding UC13 has been done manually due to some unknown problem in JMeter (no new requests were found). 
\begin{itemize}
\item UC13 regards the creation of ASOS Request to elderly people
\end{itemize}

\begin{figure}[H]
\includegraphics[width=\linewidth]{images/pendingASOS}
\caption{ UI Received ASOS requests }
\label{fig:asosrequest}
\end{figure}

The other UC14, UC15 and UC16, regarding receiving health data, sending health data and contacting emergency point, cannot be checked since the send of data cannot be done manually.

\subsection{Conclusion}
The prototype has been developed until a good point, but it still contains some bugs. In particular, many of them regards the user interface
and the website. \\
For what concerns the single goals, we can say the following:

\begin{itemize}
\item
It is possible to state that G1 has been reached. However there are some problems with the notification of third party when individual
requests are approved or rejected.

\item
A same reasoning can be applied also to G2. Indeed, UC32 and UC33 present problems with the notifications

\item
G3 has not been fully reached. To be precise, there is the problem with the constraint on 1000 individuals while accessing bulk data

\item 
G4 is achieved correctly

\item 
For what concerns G5, it has been possible to test graphically only the fact that, after having registered a user older than 60 years old, 
and when 10 seconds are elapsed, a request arrives from ASOS, that is some kind of special third party customer. \\
As mentioned in the section regarding the test of G5, the other functionalities could not be tested.\\

\par
In conclusion, the prototype is not ready to be deployed on the market due to previous problems.

\end{itemize}




\section{Installation instruction}
\subsection{Microservices}
First of all, let's check all the prerequisites needed to install and run the software:

\begin{itemize}

\item RabbitMQ is necessary, since a RabbitMQ server is needed to make the communication between microservices properly working.
It is possible to download it from there: \url{https://www.rabbitmq.com/download.html}. \\
Once the page is opened, click on the links on the right, selecting the right operative system and download the installer. 
When working with windows, it may be possible that the installation process throws an error that expresses the necessity of downloading
the Erlang programming language. In this case, just follow \url{http://www.erlang.org/downloads} and download Erlang for the version
of Windows that you are using

\item MySQL is another critical component since it has been used as a DBMS for managing persistent data. For having this working, it
is possible to follow the guide at : \url{https://beep.metid.polimi.it/documents/121843524/b5d81926-98f6-496f-bf45-a2a8e897228c} 
(ignore all the chapter regarding NetBeans and follow the ones that deal with the installation of MySQL). \\
Once that the server has been set up, it is necessary to create the databases and its admin: to do that, first of all, go to the MySQL 
folder and login with root. After that, write "create user 'trackmeadmin'@'\%' identified by 'datecallmeeting95';". \\
Now, the admin for the database has been created, and it is necessary to create all the databases used by the various microservices. \\
In order to accomplish this, run "create database share\_data\_db;", "create database group\_request\_db;", 
"create database individual\_request\_db", "create database account\_service\_db;". \\
Finally, for each database created, it is necessary to run this command: "grant all on db\_name.* to 'trackmeadmin'@'\%';"
where db\_name has to be substituted with share\_data\_db, group\_request\_db, account\_service\_db and individual\_request\_db. 

\end{itemize}

Once that all the previous steps have been accomplished, and once you have downloaded all the jars from the links present in the front page
of this document, start MySQL server (following the guide mentioned above), and after that start the RabbitMQ server. In order to do that, 
navigate to that folder in which RabbitMQ have been installed, search for "rabbitmq-server" and launch it (if you prefer other ways to launch
the server you may check the link provided before: indeed, the links to the installation instruction of the various operative systems,
provide also information on how to set up the server). \\
After that, it is possible to start the downloaded jars by means of "java -jar jar\_name". 
It is necessary to start first of the service registry, and then all the other services (and, of course, also the API gateway), in the
order you prefer: doing this should also speed up the start. 
The port that are needed to be free are the following: 8443 (API gateway), 8761 (service registry), 8089 (share data service), 8081 (group
request service), 8082 (individual request service). However, if some reasons one of these ports is busy on the
machine is which the jar will be ran, it is possible to modify this by adding "--server.port PORT\_NUMBER", while launching the jar. \\
Furthermore, is also possible to modify any of the settings that are located within the application.properties of the various project.
One can find these files in the various folders of the source code, within the folder "main". However, in order to avoid useless errors
we suggest to not modify critical parts, such as the routes used by Zuul (API gateway settings). \\

\par 
For running the tests, it is necessary to start again the RabbitMQ server and the MySQL server. After that, it is possible to go into
the root of interested project and run "mvn test". \\
Keep in mind that running the tests may take some time, due to the fact that some tests require to launch the entire spring application,
or to connect to the RabbitMQ server, or to load data into the database. 

\par 
Here it follows a list of further considerations: 
\begin{itemize}
\item If, for some reason, it is necessary to open the source code in some IDE, keep in mind that it
may be necessary to download the Lombok plugin. \\
In the case in which you are using Intellij, this link may be useful:\\
\url{https://projectlombok.org/setup/intellij}

\item Another thing that one may take into consideration, is the fact that when downloading the source code, one may have problem with too long
paths when extracting the folder. This is a problem that have been encountered only extracting the zip, and that is fixed
by using tar.gz format. 

\item When running tests that are provided with the source code, remember to not having started the application as a whole, or at least not
the one that involves the test that is going to start. This is because the same service which is running could steal the messages from the test execution environment and, therefore, conflicts can happen, leading to some bugs. \\
Furthermore, there is a timeout set to 1 minute, regarding the exchange of messages in the message queue. Keep in mind that you are probably
running an entire architecture on a single machine, therefore don't get surprised if things get stuck a bit. However, we didn't encountered
many problems in this sense.

\item If you are opening the source code in your IDE (e.g. IntelliJ) and you run the program from there, you should also configure
the databases inside the IDE. In this sense, since all the microservices are different projects, it is necessary to link each database
to its related project. \\ A final remark is that the account service db is related to the API gateway; for what concerns the others, instead,
the binding is trivial.   

\item Due to setting of the application.properties file, every time that the a jar is launched the database will be created and therefore
it will need to be populated. A little exception holds for what concerns the API gateway: indeed a SQL script named "data.sql" will be
executed loading a table that contains information regarding the available API. \\ 
Anyway, if for some reasons you don't like this option, what you need to change is: spring.jpa.hibernate.ddl-auto, from create to update.

\item 
\end{itemize}

\par
In order to run the test plans of JMeter it is necessary to have JMeter installed. Then since RabbitMQ queues has to be purged and the DBs has to be deleted, it is necessary to add some libraries:
\begin{itemize}
\item mysql-connector.jar (\url{https://dev.mysql.com/downloads/connector/j/}) to be moved into the lib folder of JMeter
\item amqp-client-3.x.x.jar (\url{https://www.rabbitmq.com/java-client.html} to be moved into the lib folder of JMeter; remember to download a version starting with 3, e.g. 3.6.6 (\url{http://repo1.maven.org/maven2/com/rabbitmq/amqp-client/})
\item Download also the project \url{https://github.com/jlavallee/JMeter-Rabbit-AMQP} and build it with Ant ("brew install ant" for Mac users). Then move the generated jar file, which can be found in /target/dist, into the lib/ext folder of JMeter.
\end{itemize}
After this configuration, it is possible to run the test plans of JMeter on GUI/non-GUI mode.

\subsection{Android application}
TODO MATTIA: Take in mind also the case in which the other group does not have any android device, therefore add notes on the simulator
and how to start things from there


Regarding the tests of the health service, all the tests should be launched on an android simulator with some tweak configurations to do:
\begin{itemize}
\item Enable the developer settings by tapping a lot of time on the build settings of the simulator.
\item Enable the mock position which can be found in the developer settings.
\end{itemize}
Moreover, the test regarding the bluetooth does not exist due to complexity: the simulator does not have the capacity to simulate bluetooth connections. Instead, there is a stub application bluetooth client which can be used to test if the communication between bluetooth devices is working as expected. Therefore, it is necessary to have two devices supporting bluetooth which are paired. Then after starting the health service on the device containing TrackMe application, a bluetooth server is running in background waiting for other devices. Now after starting the other application, it should be possible to insert the name of the device running the server and establish the connection. Once the connection is established it is possible to simulate the sending operation of health data.




\newpage

\begin{thebibliography}{9}

\bibitem{androidELS}
Android emergency location service, URL \url{https://crisisresponse.google/emergencylocationservice/how-it-works/}

\bibitem{httpstool}
Working with certificates and SSL, URL \url{https://docs.oracle.com/cd/E19830-01/819-4712/ablqw/index.html}

\bibitem{spring}
Spring framework, URL \url{https://spring.io}

\bibitem{rabbitMQ}
RabbitMQ, URL \url{https://www.rabbitmq.com}

\bibitem{rabbit-concepts}
RabbitMQ concepts about exchanges and queues, URL \url{https://www.rabbitmq.com/tutorials/amqp-concepts.html}

\end{thebibliography}

\end{document}
