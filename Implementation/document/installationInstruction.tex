\section{Installation instruction}
\subsection{Microservices}
First of all, let's check all the prerequisites needed to install and run the software:
\begin{itemize}

\item RabbitMQ is necessary, since a RabbitMQ server is needed to make the communication between microservices properly working.
It is possible to download it from there: \url{https://www.rabbitmq.com/download.html}. \\
Once the page is opened, click on the links on the right, selecting the right operative system and download the installer. 
When working with windows, it may be possible that the installation process throws an error that expresses the necessity of downloading
the Erlang programming language. In this case, just follow \url{http://www.erlang.org/downloads} and download Erlang for the version
of Windows that you are using

\item MySQL is another critical component since it has been used as a DBMS for managing persistent data. For having this working, it
is possible to follow the guide at : \url{https://beep.metid.polimi.it/documents/121843524/b5d81926-98f6-496f-bf45-a2a8e897228c}. \\
Once that the server has been set up, it is necessary to create the databases and its admin.
Open 


\end{itemize}

\subsection{Android application}
TODO TANG E MATTIA