\section{Installation instruction}
\subsection{Microservices}
First of all, let's check all the prerequisites needed to install and run the software:

\begin{itemize}

\item RabbitMQ is necessary, since a RabbitMQ server is needed to make the communication between microservices properly working.
It is possible to download it from there: \url{https://www.rabbitmq.com/download.html}. \\
Once the page is opened, click on the links on the right, selecting the right operative system and download the installer. 
When working with windows, it may be possible that the installation process throws an error that expresses the necessity of downloading
the Erlang programming language. In this case, just follow \url{http://www.erlang.org/downloads} and download Erlang for the version
of Windows that you are using

\item MySQL is another critical component since it has been used as a DBMS for managing persistent data. For having this working, it
is possible to follow the guide at : \url{https://beep.metid.polimi.it/documents/121843524/b5d81926-98f6-496f-bf45-a2a8e897228c} 
(ignore all the chapter regarding NetBeans and follow the ones that deal with the installation of MySQL). \\
Once that the server has been set up, it is necessary to create the databases and its admin: to do that, first of all, go to the MySQL 
folder and login with root. After that, write "create user 'trackmeadmin'@'\%' identified by 'datecallmeeting95';". \\
Now, the admin for the database has been created, and it is necessary to create all the databases used by the various microservices. \\
In order to accomplish this, run "create database share\_data\_db;", "create database group\_request\_db;", 
"create database individual\_request\_db", "create database account\_service\_db;". \\
Finally, for each database created, it is necessary to run this command: "grant all on db\_name.* to 'trackmeadmin'@'\%';"
where db\_name has to be substituted with share\_data\_db, group\_request\_db, account\_service\_db and individual\_request\_db. 

\end{itemize}

Once that all the previous steps have been accomplished, and once you have downloaded all the jars from the links present in the front page
of this document, start MySQL server (following the guide mentioned above), and after that start the RabbitMQ server. In order to do that, 
navigate to that folder in which RabbitMQ have been installed, search for "rabbitmq-server" and launch it (if you prefer other ways to launch
the server you may check the link provided before: indeed, the links to the installation instruction of the various operative systems,
provide also information on how to set up the server). \\
After that, it is possible to start the downloaded jars by means of "java -jar jar\_name". 
It is necessary to start first of the service registry, and then all the other services (and, of course, also the API gateway), in the
order you prefer: doing this should also speed up the start. 
The port that are needed to be free are the following: 8443 (API gateway), 8761 (service registry), 8089 (share data service), 8081 (group
request service), 8082 (individual request service). However, if some reasons one of these ports is busy on the
machine is which the jar will be ran, it is possible to modify this by adding "--server.port PORT\_NUMBER", while launching the jar. \\
Furthermore, is also possible to modify any of the settings that are located within the application.properties of the various project.
One can find these files in the various folders of the source code, within the folder "main". However, in order to avoid useless errors
we suggest to not modify critical parts, such as the routes used by Zuul (API gateway settings). \\

\par 
For running the tests, it is necessary to start again the RabbitMQ server and the MySQL server. After that, it is possible to go into
the root of interested project and run "mvn test". \\
Keep in mind that running the tests may take some time, due to the fact that some tests require to launch the entire spring application,
or to connect to the RabbitMQ server, or to load data into the database. 

\par 
Here it follows a list of further considerations: 
\begin{itemize}
\item If, for some reason, it is necessary to open the source code in some IDE, keep in mind that it
may be necessary to download the Lombok plugin. \\
In the case in which you are using Intellij, this link may be useful:\\
\url{https://projectlombok.org/setup/intellij}

\item Another thing that one may take into consideration, is the fact that when downloading the source code, one may have problem with too long
paths when extracting the folder. This is a problem that have been encountered only extracting the zip, and that is fixed
by using tar.gz format. 

\item When running tests that are provided with the source code, remember to not having started the application as a whole, or at least not
the one that involves the test that is going to start. This is because the same service which is running could steal the messages from the test execution environment and, therefore, conflicts can happen, leading to some bugs. \\
Furthermore, there is a timeout set to 1 minute, regarding the exchange of messages in the message queue. Keep in mind that you are probably
running an entire architecture on a single machine, therefore don't get surprised if things get stuck a bit. However, we didn't encountered
many problems in this sense.

\item If you are opening the source code in your IDE (e.g. IntelliJ) and you run the program from there, you should also configure
the databases inside the IDE. In this sense, since all the microservices are different projects, it is necessary to link each database
to its related project. \\ A final remark is that the account service db is related to the API gateway; for what concerns the others, instead,
the binding is trivial.   

\item Due to setting of the application.properties file, every time that the a jar is launched the database will be created and therefore
it will need to be populated. A little exception holds for what concerns the API gateway: indeed a SQL script named "data.sql" will be
executed loading a table that contains information regarding the available API. \\ 
Anyway, if for some reasons you don't like this option, what you need to change is: spring.jpa.hibernate.ddl-auto, from create to update.

\item 
\end{itemize}

\par
In order to run the test plans of JMeter it is necessary to have JMeter installed. Then since RabbitMQ queues has to be purged and the DBs has to be deleted, it is necessary to add some libraries:
\begin{itemize}
\item mysql-connector.jar (\url{https://dev.mysql.com/downloads/connector/j/}) to be moved into the lib folder of JMeter
\item amqp-client-3.x.x.jar (\url{https://www.rabbitmq.com/java-client.html} to be moved into the lib folder of JMeter; remember to download a version starting with 3, e.g. 3.6.6 (\url{http://repo1.maven.org/maven2/com/rabbitmq/amqp-client/})
\item Download also the project \url{https://github.com/jlavallee/JMeter-Rabbit-AMQP} and build it with Ant ("brew install ant" for Mac users). Then move the generated jar file, which can be found in /target/dist, into the lib/ext folder of JMeter.
\end{itemize}
After this configuration, it is possible to run the test plans of JMeter on GUI/non-GUI mode.

\subsection{Android application}
TODO MATTIA: Take in mind also the case in which the other group does not have any android device, therefore add notes on the simulator
and how to start things from there


Regarding the tests of the health service, all the tests should be launched on an android simulator with some tweak configurations to do:
\begin{itemize}
\item Enable the developer settings by tapping a lot of time on the build settings of the simulator.
\item Enable the mock position which can be found in the developer settings.
\end{itemize}
Moreover, the test regarding the bluetooth does not exist due to complexity: the simulator does not have the capacity to simulate bluetooth connections. Instead, there is a stub application bluetooth client which can be used to test if the communication between bluetooth devices is working as expected. Therefore, it is necessary to have two devices supporting bluetooth which are paired. Then after starting the health service on the device containing TrackMe application, a bluetooth server is running in background waiting for other devices. Now after starting the other application, it should be possible to insert the name of the device running the server and establish the connection. Once the connection is established it is possible to simulate the sending operation of health data.


