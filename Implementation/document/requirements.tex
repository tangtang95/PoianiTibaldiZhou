% !TeX spellcheck = en_US
\section{Requirements and functions implemented}
As already mentioned, in this section it is possible to find information regarding the requirements and the functions that are actually implemented with some motivations.

\subsection{Core requirements and functions}
All the core requirements from R1 to R9 have been implemented, since, as the name suggests, they are fundamental. 

\subsection{Data4Help requirements and functions}
All the requirements related to G1, G2, G8, G9, G12 and G13 have been implemented, basically because they 
are considered as the main features and components of Data4Help. 
This is true for all the mentioned goals expects for what concerns G9, that is the blocking of third party
customers, that can be considered as an additional functions, but can always be useful for engaging future users.
The excluded ones are the following
\begin{itemize}
\item G7, that is the management of elapsing requests 
\item G14, that is the subscription to non existing data
\end{itemize}
The motivation that stands behind this choice is basically a constraint on development time and the
fact that these were considered more as a nice feature to have, and not something really critically. 
Indeed, it is still possible to have a good prototype of the service, even without these features.
It should be noted, however, that they do not require big efforts and can be easily integrated into the
project in a second time: in particular, G7 can be considered as a periodical task to be scheduled that
checks and manages the time stamp of the pending requests. 
For what concerns G14, instead, the discussion is a bit more complex, but it basically consists in introducing a new status for the
requests and a task that operates with the requests that are in such status and that performs some checks w.r.t. to the provided dates 
(that are the starting date and the ending date of a request: this will be clear inspecting the source code of the entities that regard the
requests). \\
A final note on the requirements of Data4Help is the following and it regards the type of aggregated data that third parties can request.  
All the filters mentioned in the design document have been developed, from a server-side point of view. 
For clarifying this statement, that may sound obscure, consider that every available filter, except the one that regards GPS position data, is
only related to some plain input that a third party customer inserts in the application and that is sent to the system.
However, for filtering on GPS position a third party should define the interested region by specifying the coordinates that specify the bounds
of the interested region. 
Since, of course, this is totally no user friendly, the application could provide a list of possible cities and region and automatically
translates it. Of course, this feature could also be deployed directly on the server. \\
Nevertheless, up to now, what is present is the possibility of inserting group requests specifying the GPS filters, but this has not been 
implemented in the mobile application for the discussed reason, and, because, at this stage, it is considered sufficient to have all the 
other filters. 

\subsection{AutomatedSOS}
The goals, and the related requirements, that involves AutomatedSOS are G3 and G4. 
However, in this case, not many of the requirements have been implemented, also due to some external limitations: indeed, considering
the Android system and hardware, it is not possible to just intercept the phone call and access the microphone (for automating the phone call)
and the speakers (to retrieve the response). 
Therefore, in order to fully develop the requirements it would be necessary to use VoIP api, that requires some sort of payments.
However, some requirements such as R12, R13, R15 and R17 have been implemented anyway, with the difference that the VoIP calls are mocked
with normal phone calls and no automation is present for interacting with the emergency room operator. 
This has been considered enough for a viable product also because it has been chosen to give more importance to the core business of the 
company, that is considered to be Data4Help. 
Another few words should be spent on R17, that is "during phone calls, the GPS is set on high precision": this is automatically performed
by the Android system when calling emergency services \cite{androidELS}. \\
Finally, for what concerns the part regarding the 5 seconds the following shrewdness have been adopted: the maxitum timeout for retrieving 
the position takes in the worst case 1 second. 
After that, the more critical part is parsing a JSON that contains the emergency numbers of the various countries, but this weights only
120kB. 
It is thought that this should be fast enough to satisfies the requirements of the five seconds. 
However, the process speed of code is a problem always related with the performance of the user device and the consumed resources of other 
applications, therefore being 100\% sure that the task is accomplished within that time is impossible. Nevertheless, some tests are perfomed
and the avarage speeds satisfies this requirement.

\subsection{Non functional requirements}
For what concerns the non function requirements, as mentioned in the Design document, a microservices architecture has been developed. 
In particular, the main implemented features of the architecture are the following: 
\begin{itemize}
\item Communication between microservices, since this is crucial in order to have a viable product. Indeed, without this, almost no
requirement could be fully accomplished
\item API Gateway that performs also authentication and authorization: this is also a core feature, since it is the entry point for accessing
all the Rest API that the various microservices provide
\item Service registry, because otherwise, one should have set up all the network communication in a more static way, and also the management
of forwarding requests from the gateway would have been more complex 
\end{itemize}
The load balancer has not been implemented because, among the functions mentioned above is the last one that should be considered, since
it is very complex to have one if you an API gateway and a service registry is not present. \\

\par Some basic security feature have been developed also: indeed, all the passwords are stored in the database with bcrypt and the only type
of communication allowed between clients and the API gateway, is HTTPS. In order to do this, a custom SSL certificate has been
generated using the java keytool \cite{httpstool}.
\\
The authentication have been implemented by means of an UUID random token and the APIs has been secured in such a way that a certain client
can access only the method that he should access: for example, a user can't access methods that regards a third party customer, but, also
a user can't access API that regards another user (i.e. a user can access only his pending request, and not the pending requests of any
user) \\
As one can claim, using UUID random tokens is for sure not the best way of achieving authentication, however, the code has been designed
in such a way that is easily possible to upgrade this to the usage of JWT just implementing a single interface. This choice
has been done in order to simplify a bit the security issue and to focus more on the business core of the project. 

\subsection{Other comments}
The database regarding the collected health and position data, has not been deployed on the cloud at this
early stage, because the integration was not considered necessary. Indeed, a viable prototype can be 
exists even without this. Indeed, it is more something that regards the deploy, rather than the 
implementation itself. \\
The same reasoning has been applied also for the deployment of JARs in docker containers.